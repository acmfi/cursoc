%
% SECCI�N: INTRODUCCI�N
%
\section{Introducci�n}


En este cap�tulo vamos a ver c�mo es el modelo de compilaci�n en C, es decir, por qu� 
fases pasa nuestro c�digo desde que lo editamos hasta que obtenemos un fichero ejecutable.
En principio uno podr�a pensar que conocer lo que ocurre ``por debajo'' no es necesario
a  la hora de programar C, y de hecho no estar�a muy equivocado. Sin embargo hay ciertos
procesos que ocurren cuando compilamos nuestro c�digo C que como programadores debemos
conocer.

\begin{flushleft}
El proceso de compilaci�n de c�digo C consta de tres fases principales:
\end{flushleft}

\begin{enumerate}
	\item{\textbf{Preprocesado}}
	\item{\textbf{Compilado}}
	\item{\textbf{Enlazado}}
\end{enumerate}

Primera sorpresa: La compilaci�n es solo una fase del proceso de compilaci�n. Esta aparente
contradicci�n se explica f�cilmente: cuando hablamos del proceso de compilaci�n nos estamos
refiriendo al proceso mediante el cual transformamos nuestro c�digo fuente en un fichero
ejecutable. Sin embargo cuando nos referimos a la fase de compilaci�n nos estamos refiriendo
a la parte del proceso que se encarga de traducir c�digo fuente en instrucciones en formato
binario.

