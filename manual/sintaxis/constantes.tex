% seccion Constantes
\section{Constantes}

Las constantes son valores fijos\footnote{podr�amos decir que son ``variables constantes" 
lo que es un ox�moron, es decir, una contradicci�n, igual que ``inteligencia militar"\ldots}
 que no pueden ser modificados por el programa. Pueden ser de cualquier tipo de datos b�sico 
(punteros incluidos). Para marcar que queremos que una variable sea constante utilizamos 
la palabra reservada \verb+const+ tal que:\\
\begin{verbatim}
const int dummy = 321;         /* declaramos que dummy vale y valdr� siempre 321 */
\end{verbatim} 

No tiene demasiado sentido declarar una variable de tipo const sin darle valor inicial, 
pero algunos compiladores permiten hacerlo.

\nota{Con algunos compiladores \footnote{gcc, el compilador est�ndar de Linux, 
sin ir mas lejos}, si se intenta cambiar de valor una constante se genera un ``warning''  al compilar, pero se cambia el valor. 
C, como lenguaje para programadores que es, asume que sabes lo que est�s haciendo 
y por eso contin�a. En cualquier caso, como la constante ocupa espacio en memoria, 
siempre se podr� modificar, si \emph{realmente} se quiere ser un chico malo, 
es cuesti�n de acceder a ella directamente a base de punteros de memoria...
}

\subsection{Constantes num�ricas}

Aparte de consntantes enteras tipo \verb+234+ y en coma flotante de la forma \verb-10.4-, a veces, 
sobre todo al trabajar a bajo nivel, resulta m�s c�modo poder introducir la constante en base 8 (octal) 
o 16 (hexadecimal) que en base 10. Dado que es corriente usar estos sistemas de numeraci�n, 
C permite especificar constantes enteras en hexadecimal u octal.
Una constante hexadecimal empieza por  \verb-0x- seguido del n�mero esa base.
Igu�lmente una constante octal comienza por \verb-0-:

\begin{verbatim}
const int hex = 0x80A; /* 2058 en decimal */
const int oct = 012;  /*  10 en decimal */
\end{verbatim}

\subsection{Constantes de caracteres}

Las cadenas de caracteres se representan encerrando la cadena entre comillas dobles (\verb+"hola caracola"+ ser�a un ejemplo). Hay ciertos caracteres que no se puede (o no es c�modo)  introducir de esta forma, 
como son los caracteres de control: tabulador, retorno de carro, etc... Para introducirlos hay que 
usar unos c�digos que consisten en una barra invertida y una letra, estos son los principales:\\

\begin{table} [!h]
\begin{center}
\begin{tabular}{|c|c|}
\hline
\textbf{C�digo} & \textbf{Significado} \\ 
\hline  
        \verb-\n- & Retorno de carro  \\ 
\verb+\t+ & Tabulador\\ 
\verb+\"+ & Comillas dobles \\ 
\verb+\'+ & Comillas simples \\ 
\verb+\\ +& Barra invertida \\ 
\hline
\end{tabular} 
\caption{Constantes de Cadena}
\end{center}
\end{table}
\begin{flushleft}
Por ejemplo, este programa escribe "hola mundo" desplazado un tabulador a la izquierda y luego salta de l�nea:\\
\end{flushleft}


\ejemplo{sintaxis/ejemplo_cadenas.c}


\subsection{Constantes enumeradas}

Las constantes enumeradas permiten definir una lista de constantes representadas por identificadores.
Estas constantes son, realmente, enteros. Si no se especifica una correspondencia 
entre nombre y n�mero el compilador se encarga de asignarles n�meros correlativos 
(empezando por 0). Se pueden usar como enteros que son, pero la idea es usarlos 
en comparaciones, haciendo as� el c�digo m�s legible.\\

Para definirlas se usa la palabra reservada \verb+enum+ tal que:
\begin{verbatim}
enum color{rojo, amarillo, morado};
enum bool{false=0, true=1};
\end{verbatim} 
\consejo{Definir el tipo \texttt{bool} ayuda a conseguir c�digo m�s claro\ldots}

\begin{flushleft}
A partir de entonces se puede poner por ejemplo:
\end{flushleft}

\begin{verbatim}if(color == rojo)...\end{verbatim}
en lugar de tener que poner:
\begin{verbatim}if(color == 0)...\end{verbatim}
(�es o no es m�s claro?)
\subsection{Uso del preprocesador para definir constantes simb�licas}

Con el preprocesador, del que hablaremos m�s adelante, tambi�n se pueden definir constantes, 
aunque en este caso son realmente \textit{alias}. Al compilar se sustituye tal cual un valor por otro, 
tenga o no sentido, y no se reserva memoria para el valor. Por ejemplo, para definir que 
la longitud de algo es 10 pondr�amos:

\begin{verbatim}#define LONGITUD 10\end{verbatim}\\

\nota{Ojo, no ponemos punto y coma al final, ni signo de equivalencia. 
Hemos \emph{definido} un \emph alias para 10, que es ``LONGITUD".
Por costumbre se suelen poner las constantes as� definidas en may�sculas, 
para no confundirlas con constantes normales.}

Ejemplo de uso de const:

\ejemplo{sintaxis/ejemplo_modificadores_1.c}

%\newpage
