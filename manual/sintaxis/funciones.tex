% Seccion Funciones y subrutinas
\section{Funciones y subrutinas}

Las funciones en C desempe�an el papel de las subrutinas o
procedimientos en otros lenguajes, esto es, permiten agrupar una serie
de operaciones de tal manera que se puedan utilizar m�s tarde sin
tener que preocuparnos por c�mo est�n implementadas, simplemente
sabiendo lo que har�n. \\

El uso de funciones es una pr�ctica com�n y recomendable ya que
permite modularizar nuestro c�digo, simplificando as� el desarrollo y
la depuraci�n del mismo. Para utilizar funciones en un programa es
necesario declararlas previamente al igual que las variables (en este
caso indicaremos los argumentos de entrada y su tipo, y el tipo del
valor que devolver�) y definir las operaciones que contiene.\\

\begin{flushleft}
En C la declaraci�n de una funci�n tiene la siguiente estructura:\\
\end{flushleft}

\begin{verbatim}
tipo_devuelto nombre_funcion (argumentos);
\end{verbatim} 

\begin{flushleft}
Y su definici�n:
\end{flushleft}

\begin{verbatim}
tipo_devuelto nombre_funcion (argumentos)
{
    sentencias;
}
\end{verbatim}


\nota{A la hora de definir una funci�n que \textbf{no} acepte argumentos escribiremos
\texttt{void} en vez de los argumentos.}

\begin{flushleft}
Pongamos un ejemplo, un programa en el que queremos incluir una funci�n que devuelva el factorial de un n�mero:
\end{flushleft}


\ejemplo{sintaxis/ejemplo_funciones1.c}


\newpage
La declaraci�n
 
\begin{verbatim}
  int factorial(int a);
\end{verbatim}

debe coincidir con la definici�n de la funci�n factorial que aparece
posteriormente, si no coincide obtendremos un error en la compilaci�n del
programa. El valor que calcula la funci�n \verb+factorial()+ se devuelve por
medio de la sentencia return, �sta puede estar seguida de cualquier
expresi�n o aparecer sola. En tal caso la funci�n no devuelve ning�n
valor y al llegar a ese punto simplemente se devuelve el control a la
funci�n desde la que se invoc�.

\subsection{Paso de par�metros a funciones. Llamadas por valor}


Es importante destacar que en C todos los argumentos de una funci�n se pasan por valor. Esto es, las funciones trabajan sobre copias privadas y temporales de las variables que se le han pasado como argumentos, y no directamente sobre ellas. Lo que significa que no podemos modificar directamente desde una funci�n las variables de la funci�n que la ha llamado.\\

Veamos esto con un ejemplo:

\ejemplo{sintaxis/ejemplo_funciones2.c}


Como podemos ver, en este caso no utilizamos una variable temporal en la funci�n factorial para ir
 calcul�ndo la soluci�n, sino que vamos disminuyendo el argumento n de entrada. Esto no influye
 en la variable a (que es la que se paso como argumento a la funci�n factorial) ya que al pasarse los par�metros por valor es una copia de la variable a y no a directamente la que maneja la funci�n factorial como argumento n.\\

Si quisieramos modificar una variable llamando a una funci�n tendr�amos que pasarle como argumento a dicha funci�n la direcci�n en memoria de esa variable (un puntero a la variable). Esto lo veremos en la secci�n \ref{valor_vs_referencia}. 

%\newpage
