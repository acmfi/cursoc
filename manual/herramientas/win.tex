\section{Herramientas de desarrollo C sobre Windows}

Cuando utilizamos Windows no tenemos tan claro que compilador utilizar
ya que en Windows disponemos de herramientas portadas de Linux
(gcc.exe) o las herramientas que Microsoft nos proporciona para
desarrollar programas (Visual Studio) que podemos obtener de
Biblioteca con una licencia de estudiante de forma gratuita.

\subsection{GCC en Windows}

La compilaci�n de un programa utilizando el GCC se realiza de la misma
forma a la vista en Linux, pero debemos tener en cuenta que no todo
programa que compile en Linux compila en Windows debido a las
diferencias existentes en las llamadas al sistema. Esto es debido a  que Windows
no cumple completamente con el estandar POSIX (ver \ref{posix}) de llamadas al sistema. 
Estas
herramientas para Windows las podemos obtener por medio de diferentes
programas, ya sea bajandonos el conjunto de herramientas UNIX para
Windows \textit{MinGW}, o descarg�ndonos desde la facultad el compilador
de Ada95 \textit{(GNAT)}, el cual viene con un el GCC, ya que la
compilaci�n llevada a cabo por \textit{GNATMAKE} es una traducci�n del
lenguaje Ada a C y una posterior compilaci�n en C.  Estas herramientas
estan disponibles desde las siguientes direcciones:

\begin{itemize}
\item \url{ftp://lml.ls.fi.upm.es/pub/lenguajes/ada/gnat/3.13p/winnt/}
\item \url{http://www.mingw.org/download.shtml}
\end{itemize}

%Si nos decantamos por la utilizaci�n del GCC proporcionado por \textit{GNATMAKE} deberemos
%tener en cuenta que los programas que queramos compilar deben ser copiados a la ruta ...
Y recordad, tanto si elegimos utilizar el paquete de utilidades Linux
que encontramos en \textit{MinGW}, como si elegimos el \textit{GNAT},
deberemos incluir la ruta donde instalemos estos archivos en nuestro
\verb+PATH+ para que funcione la compilaci�n. Otra posibilidad es
crear todos los archivos dentro de la carpeta donde se encuentre
instalado \textit{MinGW} o el archivo \verb+gcc.exe+ ya que en caso
contrario, Windows no podra localizar el compilador \verb+gcc+ y nos
mostrara ese maravilloso mensaje de error:

\begin{verbatim}
"gcc" no se reconoce como un comando interno o externo, programa o
archivo por lotes ejecutable.
\end{verbatim}

% \subsection{Visual Studio}
% 
% Si decidimos utilizar como herramienta de compilaci�n este programa,
% nos encontraremos con un interfaz de desarrollo orientado a proyectos
% con C++ pero podremos utilizarlo tambien para compilar nuestros
% programas en C.
% 
% Tambi�n mencionaremos aqu� la depuraci�n disponible en Visual Studio,
% el cual nos permitira visualizar los valores de las direfentes
% variables, desde su valor en memoria y la ejecuci�n paso a paso de
% cada una de las lineas de c�digo.
% 
% \subsection{Depuracion: \texttt{gdb}}
% 
% En Windows, al igual que pasaba con el gcc, disponemos de un GDB
% portado desde Linux, el cual funciona de identica manera a la
% explicada en la secci�n \ref{gdb}.
