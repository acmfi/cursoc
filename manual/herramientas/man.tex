\section{Manuales: \texttt{man}}
\label{man}

\verb+man+\footnote{El nombre del comando viene de \texttt{manual}} es un mandato de Unix/Linux que nos informa sobre el funcionamiento de
otros mandatos.
El funcionamiento de man no s�lo se limita a informarnos sobre el uso
de mandatos, nos informa tambi�n de funciones de libreria y llamadas a
funciones del sistema. Los archivos de manual de \verb+man+ se encuentran organizados en 9 secciones, de
las cuales, ahora s�lo estamos interesados en las 3 primeras:\\

\begin{tabular}{c|l}
\textbf{Secci�n} & \textbf{Descripci�n} \\
\hline
      1 &  Programas ejecutables y comandos de la shell \\
          \hline
       2 &  Llamadas al sistema\\
           \hline
       3  & Llamadas a funciones de biblioteca \\
           \hline
\end{tabular}
\vspace{0.4cm}
%\begin{flushleft}
%Existen 6 secciones m�s pero no nos interesan para la programaci�n en C.
%\end{flushleft}

Antes de comenzar a utilizar  \verb+man+ es recomendable que nos informemos m�s
sobre su utilizaci�n (\verb+man man+).
Aqui encontraremos algunas opciones muy �tiles:\\

\begin{tabular}{c|l}
\textbf{Opci�n} & \textbf{Descripci�n} \\
\hline
     -a &  Muestra de forma consecutiva las secciones en que existe manual del comando \\
     \hline
     -k &  Muestra las paginas de manual y secciones en que se hace referencia a lo buscado \\
     \hline
\end{tabular}
\vspace{0.4cm}

En el momento de buscar informaci�n 
debemos tener en cuenta que algunas funciones y
mandatos se encuentran en varias secciones 
y, por lo tanto, deberemos indic�rselo al man antes de su ejecuci�n. Para
especificar la secci�n sobre la que queremos consultar, lo haremos de la
siguiente forma: 
\begin{verbatim}
        man [n� seccion] [mandato]
\end{verbatim}


Como ejemplo de la utilizaci�n de \verb+man+, consultemos el uso de printf como llamada a funci�n de biblioteca, mediante la siguiente linea:
\begin{verbatim}
        man 3 printf
\end{verbatim}

Ahora veamos su uso como comando de la Shell:
\begin{verbatim}
        man 1 printf
\end{verbatim}



