\section{Control Concurrente de Versiones: \texttt{cvs}}

CVS significa \textit{Concurrent Version System}. Es una herramienta
que nos permite mantener nuestro trabajo en un \textbf{repositorio},
al que m�ltiples usuarios pueden conectarse y realizar cambios. En
nuestra opini�n, es una de las mejores ayudas para el trabajo en
equipo, e incluso, para el trabajo individual (control de cambios en
las versiones). 

\subsection{Escenario de uso}

El usuario crea un repositorio en una m�quina, que almacenar� el
proyecto en el que se va a trabajar. Cada vez que se crea un archivo
nuevo en el proyecto, o un subdirectorio, se introduce en el repositorio
con un n�mero de versi�n inicial. Una vez creado el repositorio, cada usuario puede conectarse con el
servidor y descargarse una copia del mismo a su m�quina, trabajar con
los ficheros, y enviar los cambios al servidor. \\

Donde se nota realmente la potencia del sistema es cuando los
usuarios trabajan sobre los mismos ficheros. Mientras los cambios sean
en zonas diferentes de un mismo fichero, CVS se encarga de mezclar las
versiones sin interacci�n del usuario. El �nico escenario que requiere
intervenci�n del usuario es el siguiente:

\begin{itemize}
\item el usuario A se descarga la versi�n 1.3 del fichero X
\item el usuario B se descarga la versi�n 1.3 del fichero X
\item el usuario A modifica el fichero X y lo sube, generando la
  versi�n 1.4
\item el usuario B modifica el fichero X en el mismo sitio y lo sube,
  produciendo un conflicto con los cambios del usuario A
\end{itemize}

En este caso, CVS avisa al usuario B de que tiene que reconciliar a
mano ambas versiones, y le proporciona las diferencias entre sus
modificaciones y las de su compa�ero. Una vez que el usuario B corrige
a mano esos cambios, sube el fichero, generando la version 1.5.\\

CVS proporciona otras muchas ventajas, como poder generar ramas a
partir de un punto del proyecto, permitiendo que los usuarios trabajen
en ramas distintas, y muy importante, permite recuperar cualquier
versi�n de cualquier archivo existente. Podemos, por ejemplo,
recuperar un archivo tal y como estaba hace un mes.\\

A continuaci�n se muestra un ejemplo de operaci�n sobre un
repositorio:

\begin{verbatim}
Borrado gdb.tex, ya no era necesario
Cambios en las tildes de cvs.
CVS: ----------------------------------------------------------------------
CVS: Enter Log.  Lines beginning with `CVS:' are removed automatically
CVS: 
CVS: Committing in .
CVS: 
CVS: Modified Files:
CVS:    cvs.tex 
CVS: Removed Files:
CVS:    gdb.tex 
CVS: ----------------------------------------------------------------------
\end{verbatim}


\subsection{Manejo}

Os remitimos a la estupenda documentaci�n realizada por un compa�ero
de ACM en \cite{Hernando}.
