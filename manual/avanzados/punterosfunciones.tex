% Seccion Punteros a funciones
\section{Punteros a funciones}
 
\subsection{El concepto}

Esta es quiz� una de las funcionalidades que menos conoce el
programador que viene de otros lenguajes, pero que puede resultar muy
�til para resolver problemas de ``orden superior'' con una simpleza y
elegancia no muy corrientes en el paradigma imperativo.\\

Un puntero a funci�n es una variable que guarda la posici�n de memoria
de una funci�n. Por ejemplo:\\
 
\verb+void (*funcion)(void)+\\

Es un puntero a una funci�n sin argumentos, ni valor de retorno.

\subsection{�Para qu� sirven?}

Como ya vimos anteriormente, un puntero ``corriente'' nos permit�a
entre otras cosas: utilizar estructuras de datos din�micas, pasar
variables por referencia en vez de por valor.\\

Un puntero a funci�n nos permite hacer cosas m�s
``divertidas''. Podemos cargar librer�as de forma din�mica, hacer que
nuestro c�digo se automodifique en tiempo de ejecuci�n, programar
rutinas de orden superior, e implementar otra serie de funcionalidades
que amenizan nuestra tarea como desarrolladores.

\subsection{Ejemplo de orden superior}

\nota{Cuando hablamos de orden superior nos referimos a la capacidad
que tiene un lenguaje para operar con funciones como si variables
comunes se tratara. Con orden superior se puede, por ejemplo, pasar
una funci�n como argumento a otra funci�n.}

El siguiente ejemplo hace uso de la funci�n \verb+qsort()+. Dicha
funci�n reordena listas de elementos de cualquier tipo. La lista debe
ser un vector de elementos de un tama�o fijo. Por ejemplo podemos
reordenar una lista de estructuras de datos, o una simple lista de
enteros. El algoritmo quicksort es siempre el mismo. Lo �nico que
diferencia un caso de otro es el ``criterio'' de ordenaci�n de la
lista. En un caso hay que comparar dos enteros, y en el otro hay que
utilizar un criterio de comparaci�n adaptado a la estructura de
datos. Si la estructura representa la lista de empleados de una
empresa, habr� que preguntarse si queremos ordenar por nombre de
empleado, por salario o por cualquier otro criterio variopinto.\\

``El criterio'' de ordenaci�n es una funci�n que debe utilizar \verb+qsort()+
para saber c�mo ordenar las listas. Por ese motivo, \verb+qsort()+ debe
recibir un puntero a la funci�n que compare los elementos de la lista.

\ejemplo{avanzados/ejemplo_punteros_funciones.c} 

\subsection{Ejemplo de c�digo mutante}

El siguiente programa muestra un programa que se reescribe en tiempo
de ejecuci�n. Para ello primero pide 100 Bytes de memoria, despu�s
escribe en esa memoria las instrucciones correspondientes a una
funci�n, y finalmente ejecuta la funci�n con distintas variantes.

\ejemplo{avanzados/ejemplo_punteros_funciones_2.c}

El anterior ejemplo rellena unas posiciones de memoria con c�digo
m�quina. En concreto, el c�digo m�quina es equivalente a:

\begin{verbatim}   
int imprime()
{
  return 0x10;
}
\end{verbatim}   

Que en ensamblador de Intel x86 corresponde a:

\begin{verbatim}   
imprime:        
        pushl   %ebp
        movl    %esp, %ebp
        movl    $$10, %eax
        popl    %ebp
        ret          
\end{verbatim}   

y en binario Intel x86

\begin{verbatim}   
memoria[0] = 0x55;
memoria[1] = 0x89;
memoria[2] = 0xE5; 
memoria[3] = 0xB8; // mov eax <= 0x10 valor de retorno 
memoria[4] = 0x10; // 0x10
memoria[5] = 0x00; 
memoria[6] = 0x00; 
memoria[7] = 0x00;
memoria[8] = 0x5D; // ret
memoria[9] = 0xC3;
\end{verbatim}   

Para obtener el c�digo equivalente a un programa de C en ensamblador
se ha empleado: 

\begin{verbatim}   
  gcc -S fichero.c -o fichero.s # generar ensamblador
\end{verbatim}   
     
Para obtener el c�digo m�quina se ha utilizado el ensamblador:
\begin{verbatim}   
  as fichero.s -o fichero.o # generar binario
\end{verbatim}   

Y las herramientas de visualizaci�n hexadecimal
\begin{verbatim}
  hexedit fichero.o 
  hexdump -C fichero.o
\end{verbatim}   




