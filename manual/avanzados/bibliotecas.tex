\section{Gesti�n de bibliotecas de funciones}

\subsection{�Qu� es una biblioteca?}

\definicion{Biblioteca}{Consiste en un archivo binario que almacena el c�digo
compilado de funciones. }

El programa b�sico en Unix/Linux para gestionar bibliotecas es \textit{ar} (tambi�n
\textit{ranlib}) y sus funciones b�sicas son crear, modificar y
extraer funciones. Las bibliotecas m�s importantes que se utilizan en
el desarrollo de aplicaciones C son \textit{libc.a, libm.a, etc...}

\subsection{�Para qu� necesito una biblioteca?}

Las bibliotecas permiten almacenar en un mismo archivo el c�digo de muchas funciones. Las ventajas
de esta alternativa saltan a la vista:
\begin{itemize}
	\item{Cuando queramos utilizar una funci�n de la biblioteca no tendremos que buscar el c�digo
	y compilarlo, sino simplemente decirle al compilador d�nde puede encontrar la biblioteca}

	\item{Ayudan a la reutilizaci�n de c�digo, permitiendo que varios programas compartan
	porciones de c�digo.}

	\item{Promueven una programaci�n m�s modular.}
\end{itemize}



\subsection{Bibliotecas en Unix/Linux: \textit{ar}}
La sintaxis de \textit{ar} es la siguiente:

\begin{verbatim}
ar [-] [miembro] biblioteca [ficheros]
\end{verbatim}

Comentaremos las opciones mas importantes del ar:

\begin{itemize}
\item\texttt{d}: Borrar miembros de la biblioteca.
\item\texttt{m}: Mover un miembro de la biblioteca.
\item\texttt{p}: Imprimir un miembro de la biblioteca en el fichero est�ndar de salida.
\item\texttt{q}: A�adido r�pido.
\item\texttt{r}: Reemplazar ficheros en la biblioteca.
\item\texttt{t}: Mostrar una tabla con el contenido de la biblioteca.
\item\texttt{x}: Extraer miembros de la biblioteca.
\end{itemize}

\subsection{Ejemplo}

Imaginemos que tenemos una aplicaci�n para gestionar matrices, y
que tenemos una biblioteca en donde est�n incluidos ya varios ficheros
objeto llamada \textit{matriz.a}. Si ahora queremos darle mayor
funcionalidad a nuestra biblioteca de matrices y queremos a�adir
una funcion que por ejemplo haga la inversa de una matriz,
implementamos el fichero \textit{inver.c}, lo compilamos con:

\begin{verbatim}
$ gcc -c inver.c
\end{verbatim}

Esto nos crea el c�digo objeto, y ahora para a�adir esto a la biblioteca hacemos:

\begin{verbatim}
$ ar r matriz.a inver.o
\end{verbatim}

Situ�ndonos en el path donde se encuentra la biblioteca. Si el fichero
\textit{inver.o} se encuentra en otro directorio basta con hacer:

\begin{verbatim}
$ ar r matriz.a /home/micuenta/inver.o (por ejemplo)
\end{verbatim}

Para visualizar el contenido de una biblioteca:

\begin{verbatim}
$ ar t matriz.a
inver.o
multi.o
sum.o
\end{verbatim}

Si queremos m�s informaci�n:

\begin{verbatim}
$ ar tv matriz.a
rw-r--r-- 402/ 6  875 Oct 26 14:43 2003 inver.o
etc...
\end{verbatim}

Para extraer un modulo de la biblioteca:

\begin{verbatim}
$ ar xv matriz.a inver.o
\end{verbatim}
