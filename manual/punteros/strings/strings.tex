%% SECCI�N: STRINGS  (autor: ana)

\section{Strings}
\label{strings}

Los arrays de caracteres se llaman \textbf{strings}. Funcionan
exactamente igual que cualquier otro array.  En los \textbf{strings}
no hace falta una variable que guarde el tama�o del array, puesto que
se marca el final del array con el car�cter ``\verb+\0+''.  La ventaja de los
\textbf{strings} es que podemos rellenar y consultar varios elementos
del array a la vez. En lugar de:

\begin{verbatim}
cadena[0] = 'h';
cadena[1] = 'o';
cadena[2] = 'l';
cadena[3] = 'a';
cadena[4] = '\0';
\end{verbatim}

Podremos hacer esto:

\begin{verbatim}
cadena = "hola";
/* cuando metemos un texto entre comillas dobles, al copiarlo al array
   el car�cter '\0' se incluye solo. */
print ("%s",cadena);
\end{verbatim}

Una serie de fallos comunes al trabajar con strings se exponen en
\ref{error_considerar_puntero}.\\

Existen funciones espec�ficas para trabajar con strings
(\verb+#include <string.h>+) para realizar ese tipo de
tareas. Recomendamos al lector que consulte la documentaci�n de
funciones como \textit{strlen, strcpy, strncpy, strdup, strcat}.
