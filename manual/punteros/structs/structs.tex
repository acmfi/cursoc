%% SECCI�N: ESTRUCTURAS Y PUNTEROS
\section{Estructuras y punteros}

\subsection{El operador ``\texttt{->}''}
\label{operador_puntero_structs}

En programas un poco m�s elaborados, usaremos a menudo los punteros
para apuntar a estructuras (\texttt{struct}). Nada nos impide
referirnos a los campos de la estructura a la que apunta el puntero de
la forma que hemos visto hasta ahora, es decir, llamando al campo que
deseemos del contenido del puntero dado. Con este ejemplo entenderemos
mejor a qu� nos referimos:

\begin{verbatim}
struct coordenada {
   int x, y;
} coord;
struct coordenada *p_coord;
p_coord = &coord;
(*p_coord).x = 4;
(*p_coord).y = 3;
\end{verbatim}

\begin{flushleft}
Aun as� esto puede resultar algo engorroso, por eso C nos da la
posibilidad de usar el operador \texttt{->} que nos facilita las
cosas:
\end{flushleft}

\begin{verbatim}
struct coordenada {
   int x, y;
} coord;
struct coordenada *p_coord;
p_coord = &coord;
p_coord->x = 4;
p_coord->y = 3;
\end{verbatim}

Los dos c�digos anteriores ejecutan lo mismo pero en el segundo
utilizamos el operador \texttt{->}, que nos hace el c�digo m�s
legible. 

\nota{Es un error frecuente utilizar el operador punto sobre un
puntero a estructura.}  
