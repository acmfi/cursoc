%% SECCI�N: PASO POR REFERENCIA vs PASO POR VALOR

\newpage

\section{Paso por referencia vs. paso por valor}

\label{valor_vs_referencia}

\subsection{�Qu� es el paso por referencia?}

A la hora de pasar una variable como argumento a una funci�n, el paso
por referencia consiste en entregar como argumento un puntero a la
variable, y no el contenido de la variable.

\subsection{�Para qu� necesito el paso por referencia?}

Si tratamos de modificar los valores de los argumentos de una funci�n, 
estos cambios no ser�n vistos desde fuera de la misma. Esto es debido
a que los argumentos de la funci�n son copias de las variables reales y
 son almacenadas como variables locales a dicha funci�n, desapareciendo 
por tanto al regresar a la llamante. Se incluye un ejemplo a continuaci�n:

\ejemplo{punteros/referencia_valor/no_mod_args.c}

Si compilamos y ejecutamos el c�digo anterior, obtenemos el 
siguiente resultado:
\begin{verbatim}
Variable var1 = 1
Variable var1 = 1
\end{verbatim}

Como era de esperar, no se ha modificado el valor de la variable \texttt{var1}
fuera de la funci�n.

Por esto, si queremos modificar el valor de un argumento en una funci�n,
debemos pasar este par�metro \textit{por referencia}, como se muestra
en el siguiente ejemplo:

\ejemplo{punteros/referencia_valor/mod_args.c}

Como se puede comprobar, la salida de este programa es correcta:
\begin{verbatim}
Variable var1 = 1
Variable var1 = 2
\end{verbatim}

Si en una funci�n quisi�ramos modificar un par�metro que fuera un puntero, 
ser�a necesario pasar como argumento la direcci�n del mismo, es decir, 
\textit{doble indirecci�n}. Y as� sucesivamente si el argumento fuera 
doble puntero, triple puntero, etc. \\

Tambi�n es muy aconsejable el paso por referencia en el caso en que una funci�n
reciba como argumento una gran estructura de datos (arrays, matrices, ...),
puesto que el paso por valor implicar�a una copia completa de la estructura
en el espacio de memoria de la funci�n.
